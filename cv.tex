%%%%%%%%%%%%%%%%%%%%%%%%%%%%%%%%%%%%%%%%%
% Medium Length Professional CV
% LaTeX Template
% Version 2.0 (8/5/13)
%
% This template has been downloaded from:
% http://www.LaTeXTemplates.com
%
% Original author:
% Trey Hunner (http://www.treyhunner.com/)
%
% Important note:
% This template requires the resume.cls file to be in the same directory as the
% .tex file. The resume.cls file provides the resume style used for structuring the
% document.
%
%%%%%%%%%%%%%%%%%%%%%%%%%%%%%%%%%%%%%%%%%

%----------------------------------------------------------------------------------------
%	PACKAGES AND OTHER DOCUMENT CONFIGURATIONS 
%----------------------------------------------------------------------------------------
 
\documentclass[a4paper,10pt]{resume} % Use the custom resume.cls style 
\usepackage[margin=10mm]{geometry} % Document marginsTest
\usepackage[hidelinks]{hyperref}
\usepackage{xcolor}
\usepackage[T1]{fontenc}
\usepackage{mathptmx}
% your new footer definitions here
\newcommand{\tab}[1]{\hspace{.2667\textwidth}\rlap{#1}}
\newcommand{\itab}[1]{\hspace{0em}\rlap{#1}}
\name{Curriculum Vitae} % Your name 
% \address{\hspace{-13.5cm}\Large\textbf{4. CV of the researcher}} % Your address 
%\address{123 Pleasant Lane \\ City, State 12345} % Your secondary addess (optional) 
\address{Nationality: Indian \hfill Website: \href{https://aswinpvijayan.github.io}{\color{blue}{aswinpvijayan.github.io}}}

% \address{Website: \href{https://aswinpvijayan.github.io}{\color{blue}{aswinpvijayan.github.io}} \hspace{1.5cm} Email: \href{mailto:apavi@space.dtu.dk}{\color{blue}apavi@space.dtu.dk} \hfill Github: \href{https://github.com/aswinpvijayan}{\color{blue}github.com/aswinpvijayan}} % Your phone number and email

\begin{document}  
%\vspace{-0.35cm}{Email: \href{mailto:aswinpvijayan@gmail.com}{\color{blue}aswinpvijayan@gmail.com} \hspace{1.9cm}ORCiD: \href{https://orcid.org/0000-0002-1905-4194}{\color{blue}0000-0002-1905-4194} \hspace{1.7cm}Github: \href{https://github.com/aswinpvijayan}{\color{blue}github.com/aswinpvijayan}}\newline
%\hspace*{1.05cm} \href{mailto:apavi@space.dtu.dk}{\color{blue}apavi@space.dtu.dk}


\begin{rSection}{Research Interests}
Simulations of galaxy formation and evolution, high-redshift galaxies, dust in galaxies, forward modelling, synergy between simulations and observations.
\end{rSection} 

%----------------------------------------------------------------------------------------
%	WORK SECTION
%----------------------------------------------------------------------------------------

\begin{rSection}{Work Experience}
{\bf Astronomy Research Fellow} \hfill {April 2024 -- Present} 
\\
Astronomy Centre, University of Sussex, United Kingdom	
	
{\bf DAWN Postdoctoral Fellow} \hfill {Oct 2021 -- March 2024} 
\\
Cosmic Dawn Center, DTU Space, Technical University of Denmark (DTU)
\end{rSection} 

%----------------------------------------------------------------------------------------
%	EDUCATION SECTION
%----------------------------------------------------------------------------------------

\begin{rSection}{Education}

{\bf PhD in Astronomy} \hfill {30 Sept 2017 -- 30 Sept 2021}
\\ 
Astronomy Centre, University of Sussex, United Kingdom
\\
Advisors: \href{https://profiles.sussex.ac.uk/p2672-peter-thomas/about}{Prof. Peter A. Thomas} \& \href{http://stephenwilkins.co.uk/}{Dr. Stephen M. Wilkins}\\
Title: \href{https://sro.sussex.ac.uk/id/eprint/103266/}{The effect of dust in observing galaxies in the early Universe}\\
Defended thesis successfully on 4 October 2021; examiners: \href{https://romeeld.wixsite.com/romeel}{Prof. Romeel Dav{\'e}} \& \href{https://profiles.sussex.ac.uk/p91548-seb-oliver}{Prof. Seb Oliver}

{\bf MSc. Research in Astronomy, Cosmology} \hfill {1 Sept 2015 - 31 July 2017}
\\ 
Sterrewacht Leiden, Universiteit Leiden, The Netherlands\\
% MSc projects supervisors: Dr. Camila A. Correa and Dr. Jacqueline Hodge
Major Project: Analysing the impact of environment and mergers on halo concentrations,\newline supervisor \href{https://camilacorrea.com/}{Dr. Camila A. Correa.}
\\
Minor project: A search for serendipitous emission lines in \textit{ALMA} observations of high-redshift galaxies, \\supervisor \href{https://home.strw.leidenuniv.nl/~hodge/}{Dr. Jacqueline Hodge.}

{\textbf{Bachelor of Technology (B.Tech) in Engineering Physics}}  \hfill{1 Aug 2010 - 31 July 2014}\\
National Institute of Technology Calicut (NITC), Kerala, India\\
% B.Tech projects supervisor: Dr. Vari Sivaji Reddy
Bachelor Thesis: One Dimensional Organic Semiconductor Nanostructures for Solar Cell Fabrication, \\supervisor  \href{http://nitc.ac.in/index.php/?url=users/view/223/13/3}{Dr. Vari Sivaji Reddy}
%Minor in Linguistics \smallskip \\
%Member of Eta Kappa Nu \\
%Member of Upsilon Pi Epsilon \\


\end{rSection} 

%----------------------------------------------------------------------------------------
%	PUBLICATIONS SECTION
%----------------------------------------------------------------------------------------

\begin{rSection}{First \& Second Author Publications}
\begin{enumerate}
%	\item {2024 -- \bf Sub-Millimetre Galaxies with Webb: Defining the sub-mm main-sequence.} (In preparation)\\
%	Authors: \textbf{Aswin P. Vijayan} \& Minju M. Lee, and others
%	\item {2024 -- \bf Sub-Millimetre Galaxies with Webb: A framework to select sub-mm galaxy counterpart from JWST imaging.} (In preparation)\\
%	Authors: \textbf{Aswin P. Vijayan} \& Steven Gillman, and others
    \item {2023 -- \bf First Light And Reionisation Epoch Simulations (FLARES) XII: The consequences of star-dust geometry on galaxies in the EoR.} Submitted, MNRAS, \href{https://arxiv.org/abs/2303.04177}{\color{blue}arxiv:2303.04177}\\
    Authors: \textbf{Aswin P. Vijayan}, Peter A. Thomas, and others
    \item {2022 -- \bf First Light And Reionisation Epoch Simulations (FLARES) VII: The Star Formation and Metal
    Enrichment Histories of Galaxies in the early Universe.} Published, MNRAS, \href{https://arxiv.org/abs/2208.00976}{\color{blue} arXiv:2208.00976}\\
    Authors: Stephen M. Wilkins, \textbf{Aswin P. Vijayan}, and others
    \item {2022 -- \bf First Light And Reionisation Epoch Simulations (FLARES) VI: The colour evolution of galaxies z=5-15.} Published, MNRAS, \href{https://arxiv.org/abs/2207.10920}{\color{blue} arXiv:2207.10920}\\
    Authors: Stephen M. Wilkins, \textbf{Aswin P. Vijayan}, and others
    \item {2022 -- \bf First Light And Reionisation Epoch Simulations	(FLARES) V: The redshift frontier.} Published, MNRAS, \href{https://arxiv.org/abs/2204.09431}{\color{blue} arXiv:2204.09431}\\
    Authors: Stephen M. Wilkins, \textbf{Aswin P. Vijayan}, and others
    \item {2022 -- \bf First Light And Reionisation Epoch Simulations	(FLARES) III: The properties of massive dusty galaxies at cosmic dawn.} Published, MNRAS, \href{https://arxiv.org/abs/2108.00830}{\color{blue} arXiv:2108.00830}\\
    Authors: \textbf{Aswin P. Vijayan}, Stephen M. Wilkins, and others
    \item {2021 -- \bf First Light And Reionisation Epoch Simulations (FLARES) II: The Photometric Properties of High-Redshift Galaxies.} Published, MNRAS, \href{https://arxiv.org/abs/2008.06057}{\color{blue}arXiv:2008.06057}\\
    Authors: \textbf{Aswin P. Vijayan}, Christopher C. Lovell, and others
    \item {2021 -- \bf First Light And Reionisation Epoch Simulations (FLARES) I: Environmental Dependence of High-Redshift Galaxy Evolution.} Published, MNRAS, \href{https://ui.adsabs.harvard.edu/abs/2020MNRAS.tmp.3168L/abstract}{\color{blue}arXiv:2004.07283}\\
    Authors: Christopher C. Lovell, \textbf{Aswin P. Vijayan}, and others
    \item {2019 -- \bf Detailed dust modelling in the L-Galaxies semi-analytic model of galaxy formation.} Published, MNRAS, \href{https://ui.adsabs.harvard.edu/abs/2019MNRAS.489.4072V/abstract}{\color{blue}arXiv:1904.02196}\\
    Authors: \textbf{Aswin P. Vijayan}, Scott J. Clay, Peter A. Thomas, and others
\end{enumerate}
Metrics on all publications can be found from the SAO/NASA Astrophysics Data System (ADS) webpage 
\href{https://ui.adsabs.harvard.edu/public-libraries/nx11cjS2ROuxirheKfVAYw}{\color{blue}here}.
\end{rSection} 

%\begin{rSection}{Co-authored Publications}
%\begin{enumerate}
%	\item {2023 -- \bf DEIMOS spectroscopy of $z=6$ protocluster candidate in COSMOS $-$ A massive protocluster embedded in a large scale structure?} Submitted, MNRAS \\
%	Authors: Malte Brinch, Thomas R. Greve and others including \textbf{Aswin P. Vijayan}
%	\item {2023 -- \bf Size - Stellar Mass Relation and Morphology of Quiescent Galaxies at $z\ge3$ in Public $JWST$ Fields.} Submitted, ApJ \\
%	Authors: Kei Ito, Francesco Valentino, and others including \textbf{Aswin P. Vijayan}
%	\item {2023 -- \bf First Light And Reionisation Epoch Simulations (FLARES) XIV: The Balmer/4000~Å Breaks of Distant Galaxies.} Submitted, MNRAS, \href{https://arxiv.org/abs/2305.18175}{\color{blue}arxiv:2305.18175}\\
%	Authors: Stephen M. Wilkins, Christopher C. Lovell, and others including \textbf{Aswin P. Vijayan}
%	\item {2023 -- \bf First Light And Reionisation Epoch Simulations (FLARES) XIII: the Lyman-continuum emission of high-redshift galaxies.} Submitted, MNRAS, \href{https://arxiv.org/abs/2305.18174}{\color{blue}arxiv:2305.18174}\\
%	Authors: Louise T. C. Seeyave, Stephen M. Wilkins, and others including \textbf{Aswin P. Vijayan}
%	\item {2023 -- \bf Sub-Millimetre Galaxies with Webb: Near-Infrared Counterparts and Multi-wavelength Morphology.} Published, A\&A, \href{https://arxiv.org/abs/2303.17246}{\color{blue}arxiv:2303.17246}\\
%	Authors: S. Gillman, B. Gullberg, G. Brammer, \textbf{A. Vijayan} and others 
%	\item {2023 -- \bf An Atlas of Color-selected Quiescent Galaxies at $z>3$ in Public \textit{JWST} Fields.} Published, ApJ,
%	\href{https://arxiv.org/abs/2302.10936}{\color{blue}arxiv:2302.10936}\\
%	Authors: Francesco Valentino, Gabriel Brammer, and others including \textbf{Aswin P. Vijayan}
%	\item {2023 -- \bf First Light And Reionisation Epoch Simulations (FLARES) XI: [OIII] emitting galaxies at $5<z<10$.} Accepted, MNRAS, 
%	\href{https://arxiv.org/abs/2301.13038}{\color{blue}arxiv:2301.13038}\\
%	Authors: Stephen M. Wilkins, Christopher C. Lovell, \textbf{Aswin P. Vijayan}, and others
%	\item {2023 -- \bf First Light and Reionisation Epoch Simulations (FLARES) X: Environmental Galaxy Bias and Survey Variance at High Redshift.} Submitted, MNRAS,
%	\href{https://arxiv.org/abs/2301.09510}{\color{blue}arxiv:2301.09510}\\
%	Authors: Peter A. Thomas, Christopher C. Lovell, and others including \textbf{Aswin P. Vijayan}
%	\item {2023 -- \bf FLARES IX: The Physical Mechanisms Driving Compact Galaxy Formation and Evolution.} Submitted, MNRAS, \href{https://arxiv.org/abs/2301.05228}{\color{blue}arXiv:2301.05228}\\
%	Authors: William J. Roper, Christopher C. Lovell, and others including \textbf{Aswin P. Vijayan}
%	\item {2023: -- \bf Massive galaxy formation caught in action at $z\sim5$ with JWST.} Published, A\&A, 
%	\href{https://arxiv.org/abs/2212.09372}{\color{blue}arxiv:2212.09372}\\
%	Authors: Shuowen Jin, Nikolaj B. Sillassen, and others including \textbf{Aswin P. Vijayan}	 
%	\item {2022 -- \bf Dilution of chemical enrichment in galaxies 600 Myr after the Big Bang.} Accepted, Nature Astronomy,\\ \href{https://arxiv.org/abs/2212.02890}{\color{blue}arXiv:2212.02890}\\
%	Authors: Kasper E. Heintz, Gabriel B. Brammer, Clara Giménez-Arteaga, Victoria B. Strait, Claudia del P. Lagos, \textbf{Aswin P. Vijayan}, Jorryt Matthee, Darach Watson, Charlotte A. Mason, Anne Hutter, Sune Toft, Johan P. U. Fynbo, Pascal A. Oesch 
%	\item {2022 -- \bf FLARES VIII. The Emergence of Passive Galaxies in the Early Universe ($z>5$).} Submitted, MNRAS, \href{https://arxiv.org/abs/2211.07540}{\color{blue}arXiv:2211.07540}\\
%	Authors: Christopher C. Lovell, Will Roper, \textbf{Aswin P. Vijayan}, and others
%    \item {2022 -- \bf Unveiling the main sequence of galaxies at $z \ge 5$ with the James Webb Space Telescope: predictions from simulations.} Published, MNRAS, \href{https://arxiv.org/abs/2208.06180}{\color{blue}arXiv:2208.06180}\\
%    Authors: Jordan C. J. D’Silva, Claudia D. P. Lagos, and others including \textbf{Aswin P. Vijayan}
%    \item {2022 -- \bf Seeing sharper and deeper: JWST's first glimpse of the photometric and spectroscopic properties of galaxies in the epoch of reionisation.} Submitted, MNRAS, \href{https://arxiv.org/abs/2207.14265}{\color{blue} arXiv:2207.14265}\\
%    Authors: James A. A. Trussler, Nathan J. Adams, and others including \textbf{Aswin P. Vijayan}
%    \item {2022 -- \bf Discovery and properties of ultra-high redshift galaxies ($9<z<12$) in the JWST ERO SMACS 0723 Field.} Published, MNRAS, \href{https://arxiv.org/abs/2207.11217}{\color{blue}arXiv:2207.11217}\\
%    Authors: N. J. Adams, C. J. Conselice, and others including \textbf{Aswin P. Vijayan}
%    \item {2022 -- \bf The BLUETIDES mock image catalogue: Simulated observations of high-redshift galaxies and predictions for JWST imaging surveys.} Published, MNRAS, \href{https://arxiv.org/abs/2206.08941}{\color{blue} arXiv:2206.08941}\\
%    Authors: Madeline A. Marshall, Katelyn Watts, and others including \textbf{Aswin P. Vijayan}
%    \item {2022 -- \bf First Light And Reionisation Epoch Simulations	(FLARES) IV: The size evolution of galaxies at z$\ge$5.} Published, MNRAS, \href{https://arxiv.org/abs/2203.12627}{\color{blue} arXiv:2203.12627}\\
%    Authors: William J. Roper, Christopher C. Lovell, \textbf{Aswin P. Vijayan}, and others
%    \item {2022 -- \bf The Impact of Dust on the Sizes of Galaxies in the Epoch of Reionization.} Published, MNRAS, \href{https://arxiv.org/abs/2110.12075}{\color{blue} arXiv:2110.12075}\\
%    Authors: Madeline A. Marshall, Stephen Wilkins, and others including \textbf{Aswin P. Vijayan}
%    \item {2020 -- \bf Nebular Line Emission During the Epoch of Reionization.} Published, MNRAS, \href{https://ui.adsabs.harvard.edu/abs/2020MNRAS.493.6079W/abstract}{\color{blue}arXiv:1904 07504}\\
%    Authors: Stephen M. Wilkins, Christopher C. Lovell, and others including \textbf{Aswin P. Vijayan}
%\end{enumerate}
%Metrics can be found from the SAO/NASA Astrophysics Data System (ADS) webpage 
%\href{https://ui.adsabs.harvard.edu/public-libraries/nx11cjS2ROuxirheKfVAYw}{\color{blue}here}. 
%\end{rSection} 
%----------------------------------------------------------------------------------------
%	Teaching \& Work Experience SECTION
%----------------------------------------------------------------------------------------

\begin{rSection}{Teaching \& Other Experience}
{\bf Supervision of Paurush Punyasheel, MPhys student at BITS Pilani, Goa, India} \hfill {Cosmic Dawn Center, DTU}	
\\
{Dust-continuum sizes of galaxies in \textsc{Flares},}	\hfill {September-May 2023/24}
\\
{co-supervision with Prof. Thomas R. Greve}

{\bf Co-supervision of Andreas Kyster Rasmussen \& Maria Madsen, BSc students} \hfill {Cosmic Dawn Center, DTU}	
\\
{An analysis of photometric redshifts in CEERS using \texttt{eazy} \& \texttt{bagpipes},}	\hfill {September-November 2022}
\\
{co-supervision with Dr. Minju M. Lee, \href{https://stevengillman.github.io/}{Dr. Steven Gillman} \& Prof. Thomas R. Greve}

{\bf Supervision of Rebeca G Reyes Carrion, \href{https://www.dawnires.com/}{Dawn-IRES} student} \hfill {Cosmic Dawn Center, DTU}
\\
{Star formation efficiency in the First Light And Reionisation Epoch Simulations} \hfill {June-August 2022}

{\bf Co-supervision of S{\o}ren Staal, BSc student} \hfill {Cosmic Dawn Center, DTU}
\\
{Analaysing the morphology of high-z galaxies with \textsc{Flares} and \textit{JWST},} \hfill  {Feb-June 2022}
\\
{co-supervision with \href{https://stevengillman.github.io/}{Dr. Steven Gillman}}

{\textbf{2022 DAWN Fellowship Committee Member}} \hfill  {Jan-Feb 2022}
\\
{Committee of peers to select from applicants for the 2022 Dawn Postdoctoral Fellowship}

{\textbf{DISCnet Work Placement}, with Kenya Red Cross Society, Kenya (Online)} \hfill {Mar-Nov 2020}
\\
{Validation of inferences from satellite observations with ground survey data for drought impact studies}

{\bf Co-supervision of Hamish Garnett, Mphys student} \hfill {Astronomy Centre, University of Sussex}
\\
{Mass and metallicity gradients in \href{https://flaresimulations.github.io/}{\textsc{Flares}}, main supervisor Prof. Peter A. Thomas} \hfill {Oct-Apr 2019/20}
\\
{Effect of random seeds and AGN on galaxy properties in the EoR using \href{https://eagle.strw.leidenuniv.nl/}{\textsc{Eagle}}} \hfill {Oct-Apr 2020/21} 
\\
{simulation physics, main supervisor Prof. Peter A. Thomas} 

{\textbf{Associate Tutor}, School of Mathematical \& Physical Sciences, University of Sussex} \hfill {2018-2020}
\\
{Associate Tutor for the Foundation year physics course Properties of Matter (\href{http://www.sussex.ac.uk/mps/internal/departments/physicsandastronomy/modules/2020/84006}{\textcolor{blue}{F3216}})}
\\
{Associate Tutor for the third year undergraduate physics course Extragalactic Astronomy (\href{http://www.sussex.ac.uk/mps/internal/departments/physicsandastronomy/modules/2020/75497}{\textcolor{blue}{F3209}})}

\end{rSection} 

%----------------------------------------------------------------------------------------
%	Talks
%----------------------------------------------------------------------------------------

\begin{rSection}{Talks}
	{\textbf{\href{https://darkdonevski.wixsite.com/dustfocusweek}{Bridging the models \& observations of galaxies' dust in the JWST era}}, Trieste, Italy (Invited Talk)} \hfill {April 2024}
	\\
	Talk Title: \textit{Reliability of emission line ratios in the early Universe - consequences of star-dust geometry}
	
	{\textbf{\href{https://geco2023-1gyr.sciencesconf.org/resource/page/id/2}{Shedding new light on the first billion years of the Universe}}, Marseille, France} \hfill {July 2023}
	\\
	Talk Title: \textit{Reliability of Emission Line Ratios in EoR Galaxies: Impacts of Complex Star-Dust Geometry}
	
	{\textbf{\href{https://cosmicdawn.dk/meetings-and-workshops/dawn-summit-2023/}{Dawn Summit 2023}}, Cosmic Dawn Center, Denmark (Review Talk)} \hfill {June 2023}
	\\
	Talk Title: \textit{First Stars and Galaxies}	
	
	{\textbf{\href{https://lgalaxiespublicrelease.github.io/workshop2022.html}{L-Galaxies Workshop}}, University of Hertfordshire, UK (Invited Talk)} \hfill {Nov 2022}
	\\
	Talk Title: \textit{Dust modelling in L-Galaxies and some applications}	
	
	{\textbf{\href{https://cosmicdawn.dk/meetings-and-workshops/dawn-summit-2022/}{Dawn Summit 2022}}, Copenhagen, Denmark} \hfill {June 2022}
	\\
	Talk Title: \textit{Consequences of star-dust distribution and geometry on EoR galaxies}
	
	{\textbf{\href{https://events.au.dk/adam2022}{ADAM}} (Annual Danish Astronomy Meeting), Fredericia, Denmark} \hfill {May 2022}
	\\
	{Talk Title: \textit{\textsc{Flares}: Unraveling the properties of massive galaxies at cosmic dawn}}
	
	{\textbf{\href{https://nam2021.org/}{NAM}} (National Astronomy Meeting), Bath, UK (Online)} \hfill {July 2021}
	\\
	Talk title: \textit{\textsc{Flares}: The photometric properties of galaxies at cosmic dawn}
	
	{\textbf{\href{https://eas.unige.ch//EAS2021/about.jsp}{EAS}} (European Astronomical Society) Meeting (Online)} \hfill {July 2021}
	\\
	Talk title: \textit{Obscured star formation in the EoR with \textsc{Flares}}
	
	{\textbf{\href{http://sazerac-conference.org/2021/}{SAZERAC2}} (Online)} \hfill {June 2021}
	\\
	Talk title: \textit{\href{https://youtu.be/ncObY0lRN5w?list=PLp95u5tgS_YVAYXzVipf1ANzWUASqCc4Q}{\color{blue}\textit{Properties of massive dusty-galaxies at cosmic dawn}}}
	
	{\textbf{\href{http://www.virgo.dur.ac.uk/}{\textrm{Virgo Consortium Meeting}}} (Online)} \hfill {Jan 2021}
	\\
	Talk title: \textit{Photometric properties of galaxies in the \textsc{Flare} simulation}
	
	{\textbf{\href{http://sazerac-conference.org/2020/}{SAZERAC}} (Online)} \hfill {July 2020}
	\\
	Talk Title: \href{https://www.youtube.com/watch?v=S5FeQ6SUqK8}{\color{blue}\textit{Photometric properties of galaxies in the \textsc{Flare} simulation}}
	
	{\textbf{\href{https://eas.unige.ch/EAS2020/}{EAS}} (European Astronomical Society) Meeting, Leiden, The Netherlands (Online)} \hfill {Aug 2020}
	\\
	{Talk title: \textit{\textsc{Flares}: First Light And Reionisation Epoch Simulations}}
	
	{\textbf{\href{http://www.virgo.dur.ac.uk/}{Virgo Consortium Meeting}}, Durham, UK} \hfill {Jan 2020}
	\\
	{Talk title: \textit{\textsc{Flares}: First Light And Reionisation Epoch Simulations}}
	
	{\textbf{\href{http://www.virgo.dur.ac.uk/}{\textrm{Virgo Consortium Meeting}}}, Leiden, The Netherlands} \hfill {Dec 2018}
	\\
	{Talk title: \textit{Detailed dust modelling in the \textsc{L-Galaxies}’ semi-analytic model of galaxy formation}}
\end{rSection}

%----------------------------------------------------------------------------------------
%	Professional activities and organisation experience SECTION
%----------------------------------------------------------------------------------------

\begin{rSection}{Professional activities and organisation experience}
	{\textbf{co-PI of the \href{https://flaresimulations.github.io/}{\textsc{Flares}} Project}, suite of re-simulations specially designed to study } \hfill {Jan 2020-Present}\\
	galaxies observable in the high-redshift Universe
	
	{\textbf{Chair of High-Redshift galaxies: Apples to oranges from dusk to dawn}, EAS 2024, Padova} \hfill July 2024\\
	Special Session
	
	{\textbf{LOC for \href{https://d-locks.github.io/}{D-LOCKS 2024}}, at DTU Space, Denmark} \hfill January 2024
	
	{\textbf{Virgo Consortium Meeting}, Organised discussion session on Mock observations of simulations} \hfill {July 2022}
	\\
	The Virgo Consortium is an international grouping of scientists to carry out state-of-the-art cosmological \\simulations 
	
	{\textbf{\href{https://cosmicdawn.dk/event/copenhagen-dawn-conference-2022/}{DAWN Conference}}, Co-chair of `DAWN to Noon' discussion session} \hfill {June 2022}
	
	{\textbf{Astronomy on Tap} (AoT), Copenhagen, Organising committee member} \hfill {2022 -- 2023}
	
	{\textbf{Reviewer for Monthly Notices of the Royal Astronomical Society (MNRAS)}} \hfill {Oct 2021 -- Present}
	
	{\textbf{Co-organiser of DAWN caketalks}, Weekly talks at the Cosmic Dawn Center on} \hfill {Oct 2021 -- Present}
	\\
	related topics
	
	{\textbf{Scientific Organizing Committee (SOC) Member, \href{http://sazerac-conference.org/SIPS2122/1.html}{Models and Simulations of High-Redshift}}} \hfill {Oct 2021}\\
	\textbf{\href{http://sazerac-conference.org/SIPS2122/1.html}{Galaxies} - Sazerac Sip}
	\\
	{A short online conference to discuss current models and simulations of high-redshift galaxies}
	
	{\textbf{SOC Member, \href{http://sazerac-conference.org/SIPS2021/4.html}{CIDER: The Cold ISM During the Epoch of Reionisation} - Sazerac Sip}} \hfill {Feb 2021}
	\\
	{A short online conference to discuss current works on the cold ISM and dust in the EoR}
	
	{\textbf{Organizer}, Extragalactic Astronomy Journal Club, University of Sussex} \hfill {2018-2019}\\
	{Organizing the extragalactic astronomy journal club, by delegating latest papers to discuss}
\end{rSection}


%----------------------------------------------------------------------------------------
%	OUTREACH SECTION
%----------------------------------------------------------------------------------------

\begin{rSection}{Outreach}
	{\textbf{Astronomy on Tap} (AoT), Copenhagen, Denmark} \hfill {Feb 2021}
	\\
	{Talked about cosmological simulations of galaxy formation and evolution}
	
	{\textbf{BlueDot Festival}, Manchester, UK} \hfill {July 2018}
	\\
	{Volunteered at the Webb Telescope stand, talking to the public about the realm of astronomy it would be exploring}
\end{rSection}


%----------------------------------------------------------------------------------------
%	Achievements
%----------------------------------------------------------------------------------------

\begin{rSection}{Achievements}

{\textbf{\href{https://www.discnet.sussex.ac.uk/}{DISCnet}} scholarship to pursue PhD at University of Sussex, UK} \hfill {30 Sept 2017 -- 30 Sept 2021}

{Partial \textbf{Oort/NOVA} MSc scholarship from Leiden Observatory, The Netherlands} \hfill {1 Sept 2015 -- 31 July 2021}

{\href{https://en.wikipedia.org/wiki/Graduate_Aptitude_Test_in_Engineering}{\color{blue}GATE} (Graduate Aptitude Test) and \href{https://en.wikipedia.org/wiki/Joint_Entrance_Screening_Test}{\color{blue}JEST} (Joint Entrance Screening Test) in Physics,} \hfill {2015}\\
All India Rank 88 and 71 (Some of the most competitive exams for graduate schools in India)

{\href{https://en.wikipedia.org/wiki/National_Eligibility_Test}{\color{blue}NET} (National Eligibility Test, for Junior Research Fellowships) in the Physical Sciences,} \hfill {Dec 2014}\\
All India Rank 51
\end{rSection}

%----------------------------------------------------------------------------------------
%	TECHNICAL STRENGTHS SECTION
%----------------------------------------------------------------------------------------

\begin{rSection}{Computing skills}

\begin{tabular}{ @{} >{\bfseries}l @{\hspace{6ex}} l }  
Languages & \texttt{C}, \texttt{Fortran} (Basic), \texttt{Julia} (Basic), \texttt{Python}\\    
Astronomy Software & \href{https://casadocs.readthedocs.io/en/stable/}{\texttt{casa}}, \href{https://trac.nublado.org/}{\texttt{cloudy}}, \href{https://wwwmpa.mpa-garching.mpg.de/gadget/}{\texttt{gadget}}, SED fitting\\ 
Others & \LaTeX, High Performance Computing, Machine Learning, \\ & Natural Language Processing
 
\end{tabular}   

\end{rSection}

%----------------------------------------------------------------------------------------
%	CONFERENCES & Workshops
%----------------------------------------------------------------------------------------

\begin{rSection}{Attended CONFERENCES \& WORKSHOPS}
{\href{skirtdays2022.ugent.be}{\color{blue}\textrm{SKIRT Days 2022}} (Ghent University, Belgium), \href{https://lgalaxiespublicrelease.github.io/workshop2022.html}{\color{blue}\textrm{L-Galaxies Workshop}} (University of Hertfordshire, UK),} \hfill {2022}
\\
\href{http://www.virgo.dur.ac.uk/}{\color{blue}\textrm{Virgo Consortium Meeting}} (Max-Planck-Institut f\"{u}r Astrophysik, Garching, Germany),\\
\href{https://cosmicdawn.dk/event/copenhagen-dawn-conference-2022/}{\color{blue}\textrm{Dawn Conference}} (Copenhagen, Denmark), {\href{https://cosmicdawn.dk/meetings-and-workshops/dawn-summit-2022/}{\color{blue}\textrm{Dawn Summit}}} (Copenhagen, Denmark), \\{\href{https://events.au.dk/adam2022/conference}{\color{blue}\textrm{ADAM 2022}} (Federecia, Denmark)}

{\href{https://eas.unige.ch//EAS2021/about.jsp}{\color{blue}\textrm{EAS 2021}} (Online), \href{http://sazerac-conference.org/2021/}{\color{blue}\textrm{SAZERAC2}} (Online), \href{http://www.virgo.dur.ac.uk/}{\color{blue}\textrm{Virgo Consortium Meeting}} (Online)} \hfill {2021}

{\href{https://metals-dust.sciencesconf.org/}{\color{blue}\textrm{The Rise of Metals and Dust in Galaxies through Cosmic Time}} (Online), \href{https://eas.unige.ch/EAS2020/}{\color{blue}\textrm{EAS 2020}} (Online),} \hfill {2020}
\\
\href{http://sazerac-conference.org/2020/}{\color{blue}\textrm{SAZERAC}} (Online), \href{http://www.iiti.ac.in/people/~firstbillion/}{\color{blue}\textrm{International Conference on Observing The First Billion Years of the Universe
\\
Using Next Generation Telescopes}} (Indore, India), \href{http://www.virgo.dur.ac.uk/}{\color{blue}\textrm{Virgo Consortium Meeting}} (ICC, Durham, UK)

{\href{http://www-star.st-and.ac.uk/samcss/}{\color{blue}\textrm{St Andrews Monte Carlo Radiation Hydrodynamics Summer School}} (University of St Andrews, UK), } \hfill {2019}
\\
\href{https://www.sussex.ac.uk/discus/contact/stfcsummerschool2019}{\color{blue}\textrm{STFC Data Intensive, Artificial Intelligence and Machine Learning Summer School 2019}} (University of Sussex, UK)

{\href{https://events.prace-ri.eu/event/783/}{\color{blue}\textrm{Parallel and GPU Programming in Python}} (Surfsara, Amsterdam, The Netherlands), } \hfill {2018}
\\
\href{http://www.virgo.dur.ac.uk/}{\color{blue}\textrm{Virgo Consortium Meeting}} (Lorentz Centre, Leiden, The Netherlands), \href{https://indico.cern.ch/event/702529/overview}{\color{blue}\textrm{STFC Summer School in Artificial Intelligence and Machine Learning}} (University College London, UK), \href{https://www.eventbrite.co.uk/e/discnet-high-performance-computing-tickets-44362834432?utm_source=eb_email&utm_medium=email&utm_campaign=reminder_attendees_48hour_email&utm_term=eventname}{\color{blue}\textrm{DISCnet: High Performance Computing}} (Old Thorns Manor, Liphook, UK), \href{https://www.eventbrite.co.uk/e/discnet-machine-learning-tickets-44362574655?utm_source=eb_email&utm_medium=email&utm_campaign=order_confirmation_email&utm_term=eventname&ref=eemailordconf}{\color{blue}\textrm{DISCnet: Machine Learning}} (Old Thorns Manor, Liphook, UK), \href{https://www.eventbrite.co.uk/e/gradnet-observational-astrophysics-workshop-5-7-march-2018-registration-34828328468?utm_source=eb_email&utm_medium=email&utm_campaign=order_confirmation_email&utm_term=eventname&ref=eemailordconf}{\color{blue}\textrm{GRADnet: Observational Astrophysics Workshop}} (Old Thorns Manor, Liphook, UK), \href{hhttps://www.eventbrite.co.uk/e/discnet-statistics-and-data-analysis-tickets-41817533368?utm_source=eb_email&utm_medium=email&utm_campaign=order_confirmation_email&utm_term=eventname&ref=eemailordconf}{\color{blue}\textrm{DISCnet: Statistics and Data Analysis}} (Old Thorns Manor, Liphook, UK)

{\href{http://www.virgo.dur.ac.uk/}{\color{blue}\textrm{Virgo Consortium Meeting}} (Max-Planck-Institut f\"{u}r Astrophysik, Garching, Germany), \href{https://www.eventbrite.co.uk/e/discnet-introduction-to-big-data-tickets-38725058693?utm_source=eb_email&utm_medium=email&utm_campaign=order_confirmation_email&utm_term=eventname&ref=eemailordconf}{\color{blue}\textrm{DISCnet: }}} \hfill {2017}
\\
\href{https://www.eventbrite.co.uk/e/discnet-introduction-to-big-data-tickets-38725058693?utm_source=eb_email&utm_medium=email&utm_campaign=order_confirmation_email&utm_term=eventname&ref=eemailordconf}{\color{blue}\textrm{Introduction to Big Data}} (Old Thorns Manor, Liphook, UK), \href{https://www.eventbrite.co.uk/e/discnet-software-carpentry-and-public-engagement-with-research-training-tickets-38308865849?utm_source=eb_email&utm_medium=email&utm_campaign=order_confirmation_email&utm_term=eventname&ref=eemailordconf}{\color{blue}\textrm{DISCnet: Software Carpentry and Public Engagement with Research Training}} (Old Thorns Manor, Liphook, UK)
\end{rSection} 


%----------------------------------------------------------------------------------------
%	CONFERENCES & Workshops
%----------------------------------------------------------------------------------------

%\begin{rSection}{Internships}
%{\textbf{Indian Academy of Science Summer Research Fellow}} \hfill {April-June 2013}
%\\
%Worked on the topic Our Universe and its Cosmological Evolution, under \href{https://www.ctp-jamia.res.in/people/aasen.html}{Prof. Anjan Ananda Sen}, Central for Theoretical Physics (CTP), Jamia Millia Islamia (JMI), New Delhi.
%
%{\textbf{Summer Internship}} \hfill {May-June 2012}
%\\
%Studied the technological aspects of creating Bose-Einstein Condensation of Rubidium atoms mainly focusing on the vacuum technologies used, under \href{http://sites.iiserpune.ac.in/~umakant.rapol/}{Prof. Umakant Rapol}, Indian Institute of Science Education and Research (IISER), Pune
%\end{rSection}

\begin{rSection}{Languages}
Malayalam (Native), English (Fluent), Hindi (Intermediate), Sanskrit (Intermediate)
\end{rSection}

\begin{rSection}{References}
\begin{itemize}
   \item \href{http://stephenwilkins.co.uk/}{Prof. Stephen M. Wilkins}, Astronomy Centre, University of Sussex, Brighton, UK\\
   \textbf{Email}: s.wilkins@sussex.ac.uk
   \item \href{https://cosmicdawn.dk/staff/thomas/}{Prof. Thomas R. Greve}, Cosmic Dawn Center, DTU Space, DTU, Kongens Lyngby, Denmark\\
   \textbf{Email}: tgreve@space.dtu.dk
   \item \href{https://profiles.sussex.ac.uk/p2672-peter-thomas/about}{Prof. Peter A. Thomas}, Astronomy Centre, University of Sussex, Brighton, UK\\
   \textbf{Email}: P.A.Thomas@sussex.ac.uk
   \item \href{https://cosmicdawn.dk/staff/georgios-magdis/}{Prof. Georgios E. Magdis}, Cosmic Dawn Center, DTU Space, DTU, Kongens Lyngby, Denmark\\
   \textbf{Email}: geoma@space.dtu.dk
\end{itemize}
\end{rSection}
\end{document}
